\chapter{The Conic Sections}
Throughout this chapter, the following variables will represent special values concerning conic sections.
\begin{table}[H]
\caption{Special Values in Conic Sections}
\centering
\begin{tabular}{c c}
\hline
$a$ & The length of the semi-major axis.\footnote{($\frac{1}{2}$ of the length of the major axis)}\\
$b$ & The length of the semi-minor axis.\footnote{($\frac{1}{2}$ of the length of the minor axis)}\\
$h$ & The x-offset of the conic from the origin.\\
$k$ & The y-offset of the conic from the origin.\\

\hline
\end{tabular}
\label{table:SpecValsConics}
\end{table}

\section{The Circle}
\begin{defn}[The Standard Form for an Equation of the Circle]
$\displaystyle{(x-h)^{2}+(y-k)^{2}=c^{2}}$
\end{defn}
In the circle, the length of the semi-major axis (and by extension the semi-minor axis) equivalent to as the radius. Thus, this form is also known as the \emph{Center Radius Form}. In this equation the coordinates $(h,k)$ represent the center of the circle and $c$ represents the radius. In light of this, a circle can be considered a special case of the Ellipse.

\begin{defn}[The Expanded Form of an Equation for the Circle]
\begin{gather*}
\displaystyle{ax^{2}+by^{2}+cx+dy+e=0}\\
\text{where $a=b$ and $\sign{a}=\sign{b}=\sign{c}$}
\end{gather*}
\end{defn}
Note that this form of the equation can be manipulated back into the center radius form through completing the square.

\begin{subequations}
The steps below demonstrate conversion from Expanded to Center Radius Form:
    \begin{align}
        \displaystyle{ax^{2}+by^{2}+cx+dy+e=0}\\
        \displaystyle{ax^{2}+ay^{2}+cx+dy+e=0}\\
        \displaystyle{{ax^{2}+ay^{2}+cx+dy=-e}}\\
        \displaystyle{{ax^{2}+ay^{2}+cx+dy=-e}}\\
        \displaystyle{{x^{2}+y^{2}+\frac{c}{a}x+\frac{d}{a}y=-\frac{e}{a}}}\\
        \displaystyle{\left(x^{2}+\frac{c}{a}x+\left(\frac{c}{2a}\right)^{2}\right)+\left(y^{2}+\frac{d}{a}x+\left(\frac{d}{2a}\right)^{2}\right)=-\frac{e}{a}+\left(\frac{c}{2a}\right)^{2}+\left(\frac{d}{2a}\right)^{2}}\\
        \displaystyle{\left(x+\frac{c}{2a}\right)^{2}+\left(y+\frac{d}{2a}\right)^{2}=\left(-\frac{e}{a}+\left(\frac{c}{2a}\right)^{2}+\left(\frac{d}{2a}\right)^{2}\right)}
    \end{align}
\end{subequations}

\begin{subequations}
Here is a concrete example of the above process:
    \begin{align}
        4x^2+4y^2-16x-24y+51=0\\
        4x^2+4y^2-16x-24y=-51\\
        x^2+y^2-4x-6y=-\frac{51}{4}\\
        (x^2-4x+4)+(y^2-6y+9)=-\frac{51}{4}+4+9\\
        (x-2)^2+(y-3)^2=\frac{1}{4}
    \end{align}
\end{subequations}

\section{The Ellipse}
\begin{defn}[The Standard Form for an Equation of the Ellipse]
$\displaystyle{\frac{(x-h)^2}{a^2}+\frac{(y-k)^2}{b^2}=1}$
\end{defn}

\section{The Parabola}
\begin{defn}[The Standard Form for an Equation of the Parabola]
$\displaystyle{y=4ax}$
\end{defn}
\section{The Hyperbola}
\begin{defn}[The Standard Form for an Equation of the Ellipse]
$\displaystyle{\frac{(x-h)^2}{a^2}-\frac{(y-k)^2}{b^2}=1}$
\end{defn}
